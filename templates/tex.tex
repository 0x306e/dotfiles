\documentclass[uplatex, dvipdfmx, a4paper]{ujarticle}
% 必須
\usepackage[left=18.05mm, right=18.05mm, top=25.40mm, bottom=25.40mm]{geometry}
\usepackage{amsmath, amssymb}
\usepackage[dvipdfmx]{graphicx}
\usepackage[subrefformat=parens]{subcaption}
\usepackage{here}
\usepackage{bxwareki}

% ページ番号を右上へ
\usepackage{fancyhdr}
\pagestyle{fancy}
\lhead{}
\rhead{}
\rhead{\thepage{}}
\cfoot{}
\renewcommand{\headrulewidth}{0pt}

% 必要に応じて使う
% \usepackage{url}
% \usepackage{slashbox}
% \usepackage[american voltages]{circuitikz}
% \usepackage{listings}
% \lstset{
%   basicstyle={\ttfamily},
%   identifierstyle={\small},
%   commentstyle={\smallitshape},
%   keywordstyle={\small\bfseries},
%   ndkeywordstyle={\small},
%   stringstyle={\small\ttfamily},
%   frame={tb},
%   breaklines=true,
%   columns=[l]{fullflexible},
%   numbers=left,
%   xrightmargin=0zw,
%   xleftmargin=3zw,
%   numberstyle={\scriptsize},
%   stepnumber=1,
%   numbersep=1zw,
%   lineskip=-0.5ex
% }
% \renewcommand{\lstlistingname}{source code}

% si単位で書きたいとき使う
\usepackage{siunitx}
\newcommand{\sinum}[2]{\num{#1}\ \si{[#2]}}
\newcommand{\unit}[1]{\ \si{[#1]}}

% in/out
% \newcommand{\uin}[0]{_\mathrm{in}}
% \newcommand{\uout}[0]{_\mathrm{out}}

% for maths
% \newcommand{\x}[0]{$x$}
% \usepackage{bm}
% \usepackage{gauss}
% \newcommand{\rank}{\mathop{\rm rank}\nolimits}
% \newcommand{\dim}{\mathop{\rm dim}\nolimits}
% \newcommand{\Ker}{\mathop{\rm Ker}\nolimits}
% \newcommand{\paren}[1]{\left(#1\right)}
% \newcommand{\pmatrixparen}[1]{\begin{pmatrix}#1\end{pmatrix}}
% \newcommand{\bmatrixparen}[1]{\begin{bmatrix}#1\end{bmatrix}}

\title{title}
\author{名前}
\date{\warekitoday}

\begin{document}
\section{実験環境}
\begin{itemize}
  \item 室温: $\si{\degreeCelsius}$
  \item 湿度: $\%$
\end{itemize}

\section{実験目的}

\section{使用機器}
\begin{table}[H]
  \centering
  \caption{}
  \label{tab:machinary}
  \begin{tabular}{|c|} \hline
    hoge \\
  \end{tabular}
\end{table}

\section{実験内容及びその結果}

\begin{table}[H]
  \centering
  \caption{}
  \label{tab:}
  \begin{tabular}{|c|} \hline
    hoge \\
  \end{tabular}
\end{table}

\begin{figure}[H]
  \centering
  % \includegraphics[keepaspectratio, scale=1.0]{./to/path.png}
  \caption{}
  \label{fig:}
\end{figure}

\section{検討事項}
\begin{enumerate}
  \item{hoge}
\end{enumerate}

\section{吟味}
\begin{itemize}
  \item{hoge}
\end{itemize}

\bibliographystyle{junsrt}
\bibliography{main}
\end{document}
