\documentclass[uplatex, dvipdfmx, a4paper]{ujarticle}
% 余白サイズの設定
\usepackage[left=18.05mm, right=18.05mm, top=25.40mm, bottom=25.40mm]{geometry}
% 数式で使うやつ
\usepackage{amsmath, amssymb}
% SI単位を直感的に記述できるやつ
\usepackage{siunitx}
% urlの表示統一 bibtex用
\usepackage{url}

%includegraphics するやつ
\usepackage[dvipdfmx]{graphicx}
% 図表配置の強化
\usepackage{here}
% 文字通りsubcaption
\usepackage[subrefformat=parens]{subcaption}
% ソースコードの図示とかで使う 手動インストールが必要
\usepackage{listings}
\lstset{
  basicstyle={\ttfamily},
  identifierstyle={\small},
  commentstyle={\smallitshape},
  keywordstyle={\small\bfseries},
  ndkeywordstyle={\small},
  stringstyle={\small\ttfamily},
  frame={tb},
  breaklines=true,
  columns=[l]{fullflexible},
  numbers=left,
  xrightmargin=0zw,
  xleftmargin=3zw,
  numberstyle={\scriptsize},
  stepnumber=1,
  numbersep=1zw,
  lineskip=-0.5ex
}
\renewcommand{\lstlistingname}{source code}
% 回路図描けるマン
\usepackage[american voltages]{circuitikz}
% 表中に斜線引けるやつ
% \usepackage{slashbox}

% ページのスタイル変えるやつ ページ番号を右上へ
\usepackage{fancyhdr}
\pagestyle{fancy}
\lhead{}
\rhead{}
\rhead{\thepage{}}
\cfoot{}
\renewcommand{\headrulewidth}{0pt}

% units
\newcommand{\sinum}[2]{\num{#1}\ \si{[#2]}}
\newcommand{\unit}[1]{\ \si{[#1]}}

\newcommand{\x}[0]{$x$}
\newcommand{\uin}[0]{_\mathrm{in}}
\newcommand{\uout}[0]{_\mathrm{out}}


\begin{document}
\section{実験環境}
\begin{itemize}
  \item 室温: $\si{\degreeCelsius}$
  \item 湿度: $\%$
\end{itemize}

\section{実験目的}

\section{使用機器}
\begin{table}[H]
  \centering
  \caption{}
  \label{tab:machinary}
  \begin{tabular}{|c|} \hline
    hoge \\
  \end{tabular}
\end{table}

\section{実験内容及びその結果}

\begin{table}[H]
  \centering
  \caption{}
  \label{tab:}
  \begin{tabular}{|c|} \hline
    hoge \\
  \end{tabular}
\end{table}

\begin{figure}[H]
  \centering
  % \includegraphics[keepaspectratio, scale=1.0]{./to/path.png}
  \caption{}
  \label{fig:}
\end{figure}

\section{検討事項}
\begin{enumerate}
  \item{hoge}
\end{enumerate}

\section{吟味}
\begin{itemize}
  \item{hoge}
\end{itemize}

\bibliographystyle{junsrt}
\bibliography{main}
\end{document}
